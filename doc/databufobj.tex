\subsection{DataBuf Objects}

\class{DataBuf} objects manage buffers which are used to hold data to
be sent to and received from the server.

\class{DataBuf} objects contain an embedded Sybase \code{CS_DATAFMT}
structure and allocated buffers suitable for binding the contained
data to Sybase-CT API functions.

When constructed from native Python or Sybase data types a buffer is
created for a single value.  When created using a \class{CS_DATAFMT}
object the \code{count} attribute is used to allocate buffers suitable
for array binding.  A \code{count} of zero is treated the same as
\code{1}.

The \class{DataBuf} objects have the same attributes as a
\class{CS_DATAFMT} object but the attributes which describe the memory
are read only and cannot be modified.

\begin{tabular}{l|l|l}
attribute & type & read only? \\
\hline
\code{name}      & \code{string} & no \\
\code{datatype}  & \code{int}    & yes \\
\code{format}    & \code{int}    & no \\
\code{maxlength} & \code{int}    & yes \\
\code{scale}     & \code{int}    & yes \\
\code{precision} & \code{int}    & yes \\
\code{status}    & \code{int}    & no \\
\code{count}     & \code{int}    & yes \\
\code{usertype}  & \code{int}    & yes \\
\code{strip}     & \code{int}    & no \\
\end{tabular}

In addition the \class{DataBuf} object behaves like a fixed length
mutable sequence.

Adapted from \texttt{Sybase.py}, this is how you create a set of
buffers suitable for retrieving a number of rows from the server:

\begin{verbatim}
def row_bind(cmd, count = 1):
    status, num_cols = cmd.ct_res_info(CS_NUMDATA)
    if status != CS_SUCCEED:
        raise 'ct_res_info'
    bufs = []
    for i in range(num_cols):
        status, fmt = cmd.ct_describe(i + 1)
        if status != CS_SUCCEED:
            raise 'ct_describe'
        fmt.count = count
        status, buf = cmd.ct_bind(i + 1, fmt)
        if status != CS_SUCCEED:
            raise 'ct_bind'
        bufs.append(buf)
    return bufs
\end{verbatim}

Then once the rows have been fetched, this is how you extract the data
from the buffers:

\begin{verbatim}
def fetch_rows(cmd, bufs):
    rows = []
    status, rows_read = cmd.ct_fetch()
    if status == CS_SUCCEED:
        for i in range(rows_read):
            row = []
            for buf in bufs:
                row.append(buf[i])
            rows.append(tuple(row))
    return rows
\end{verbatim}

To send a parameter to a dynamic SQL command or a stored procedure you
are likely to create a \class{DataBuf} object directly from the value
you wish to send.  For example:

\begin{verbatim}
if cmd.ct_command(CS_RPC_CMD, 'sp_help', CS_NO_RECOMPILE) != CS_SUCCEED:
    raise 'ct_command'
buf = DataBuf('sysobjects')
buf.status = CS_INPUTVALUE
if cmd.ct_param(buf)  != CS_SUCCEED:
    raise 'ct_param'
if cmd.ct_send() != CS_SUCCEED:
    raise 'ct_send'
\end{verbatim}

Note that it is your responsibility to make sure that the buffers are
not deallocated before you have finished using them.  If you are not
careful you will get a segmentation fault.
