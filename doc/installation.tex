\section{Installation}

\subsection{Prerequisites}

\begin{itemize}
\item Python 1.5.2 or later.

\item C compiler

The \module{Sybase} package contains an extension module written in C.
The extension module is used by the Python wrapper module
\module{Sybase.py}.

\item Sybase client libraries

The \module{Sybase} package uses the Sybase client libraries which
should have come as part of your Sybase installation.

If you are using Linux, Sybase provide a free version of their
database which can be downloaded from \url{http://linux.sybase.com/}.

\item mxDateTime

If the \module{mxDateTime} package is installed the \module{Sybase}
module can return \code{datetime} values as \class{DateTime} objects.
If \module{mxDateTime} is not present the module will use the
\class{DateTime} object defined in the \module{sybasect} extension
module.
\end{itemize}
\subsection{Installing}

The \module{Sybase} package uses the \module{distutils} package so all
you need to do is type the following command as root:

\begin{verbatim}
python setup.py install
\end{verbatim}

To disable bulkcopy support you should use the following commands:

\begin{verbatim}
python setup.py build_ext -U WANT_BULKCOPY
python setup.py install
\end{verbatim}

The default build does not enable threading in the extension module so
if you want threading enabled you will have to do this:

\begin{verbatim}
python setup.py build_ext -D WANT_THREADS
python setup.py install
\end{verbatim}

This is the first release which supports FreeTDS.  To compile for
FreeTDS you should use the following commands:

\begin{verbatim}
python setup.py build_ext -D HAVE_FREETDS -U WANT_BULKCOPY
python setup.py install
\end{verbatim}

To build with FreeTDS and threads use the following commands:

\begin{verbatim}
python setup.py build_ext -D WANT_THREADS,HAVE_FREETDS -U WANT_BULKCOPY
python setup.py install
\end{verbatim}

You will probably experience some segfaults with FreeTDS using the
\class{Cursor} \method{callproc()} method and when using named
arguments.

If you have problems with the installation step, edit the
\texttt{setup.py} file to specify where Sybase is installed and the
name of the client libraries.  Make sure that you contact the package
author so that the installation process can be made more robust for
other people.

\subsection{Testing}

The most simple way to test the \module{Sybase} package is to run a
test application.  The arguments to the \function{Sybase.connect()}
function are \var{server} (from the interfaces file), \var{username},
\var{password}, and \var{database}.  The \var{database} argument is
optional.  Make sure you substitute values that will work in your
environment.

\begin{verbatim}
>>> import Sybase
>>> db = Sybase.connect('SYBASE', 'sa', '')
>>> c = db.cursor()
>>> c.callproc('sp_help')
>>> for t in c.description:
...     print t
... 
('Name', 0, 0, 30, 0, 0, 0)
('Owner', 0, 0, 30, 0, 0, 32)
('Object_type', 0, 0, 22, 0, 0, 32)
>>> for r in c.fetchall():
...     print r
... 
('spt_datatype_info', 'dbo', 'user table')
('spt_datatype_info_ext', 'dbo', 'user table')
('spt_limit_types', 'dbo', 'user table')
  :
  :
('sp_prtsybsysmsgs', 'dbo', 'stored procedure')
('sp_validlang', 'dbo', 'stored procedure')
>>> c.nextset()
1
>>> for t in c.description:
...     print t
... 
('User_type', 0, 0, 15, 0, 0, 0)
('Storage_type', 0, 0, 15, 0, 0, 0)
('Length', 6, 0, 1, 0, 0, 0)
('Nulls', 11, 0, 1, 0, 0, 0)
('Default_name', 0, 0, 15, 0, 0, 32)
('Rule_name', 0, 0, 15, 0, 0, 32)
\end{verbatim}

\subsection{Installing Sybase on Linux}

There is a very nice guide to installing Sybase on Linux at Linux
Planet.  \url{http://www.linuxplanet.com/linuxplanet/tutorials/4323/1/}

(Thanks to Tim Churches for pointing this out).
